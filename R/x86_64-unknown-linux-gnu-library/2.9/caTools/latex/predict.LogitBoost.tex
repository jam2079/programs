\HeaderA{predict.LogitBoost}{Prediction Based on LogitBoost Classification Algorithm}{predict.LogitBoost}
\keyword{classif}{predict.LogitBoost}
\begin{Description}\relax
Prediction or Testing using logitboost classification algorithm
\end{Description}
\begin{Usage}
\begin{verbatim}## S3 method for class 'LogitBoost':
predict(object, xtest, type = c("class", "raw"), nIter=NA, ...)\end{verbatim}
\end{Usage}
\begin{Arguments}
\begin{ldescription}
\item[\code{object}] An object of class "LogitBoost" see "Value" section of 
\code{\LinkA{LogitBoost}{LogitBoost}} for details
\item[\code{xtest}] A matrix or data frame with test data. Rows contain samples 
and columns contain features
\item[\code{type}] See "Value" section
\item[\code{nIter}] An optional integer, used to lower number of iterations 
(decision stumps) used in the decision making. If not provided than the 
number will be the same as the one provided in \code{\LinkA{LogitBoost}{LogitBoost}}. 
If provided than the results will be the same as running 
\code{\LinkA{LogitBoost}{LogitBoost}} with fewer iterations. 
\item[\code{...}] not used but needed for compatibility with generic predict 
method
\end{ldescription}
\end{Arguments}
\begin{Details}\relax
Logitboost algorithm relies on a voting scheme to make classifications. Many
(\code{nIter} of them) week classifiers are applied to each sample and their
findings are used as votes to make the final classification. The class with 
the most votes "wins". However, with this scheme it is common for two cases 
have a tie (the same number of votes), especially if number of iterations is 
even. In that case NA is returned, instead of a label.
\end{Details}
\begin{Value}
If type = "class" (default) label of the class with maximal probability is 
returned for each sample. If type = "raw", the a-posterior probabilities for 
each class are returned.
\end{Value}
\begin{Author}\relax
Jarek Tuszynski (SAIC) \email{jaroslaw.w.tuszynski@saic.com}
\end{Author}
\begin{SeeAlso}\relax
\code{\LinkA{LogitBoost}{LogitBoost}} has training half of LogitBoost code
\end{SeeAlso}
\begin{Examples}
\begin{ExampleCode}# See LogitBoost example\end{ExampleCode}
\end{Examples}

