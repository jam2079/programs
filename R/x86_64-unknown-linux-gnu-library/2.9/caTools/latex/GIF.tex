\HeaderA{read.gif \& write.gif}{Read and Write Images in GIF format}{read.gif .Ramp. write.gif}
\aliasA{read.gif}{read.gif \& write.gif}{read.gif}
\aliasA{write.gif}{read.gif \& write.gif}{write.gif}
\keyword{file}{read.gif \& write.gif}
\begin{Description}\relax
Read and write files in GIF format. Files can contain single images
or multiple frames. Multi-frame images are saved as animated GIF's.
\end{Description}
\begin{Usage}
\begin{verbatim}
read.gif(filename, frame=0, flip=FALSE, verbose=FALSE) 
write.gif(image, filename, col="gray", scale=c("smart", "never", "always"), 
    transparent=NULL, comment=NULL, delay=0, flip=FALSE, interlace=FALSE)
\end{verbatim}
\end{Usage}
\begin{Arguments}
\begin{ldescription}
\item[\code{filename}] Character string with name of the file. In case of 
\code{read.gif} URL's are also allowed.
\item[\code{image}] Data to be saved as GIF file. Can be a 2D matrix or 3D array. 
Allowed formats in order of preference:
\Itemize{
\item array of integers in [0:255] range - this is format required by GIF 
file, and unless \code{scale='always'}, numbers will not be rescaled.
Each pixel \code{i} will have associated color \code{col[image[i]+1]}. 
This is the only format that can be safely used with non-continuous color 
maps.
\item array of doubles in [0:1] range - Unless \code{scale='never'} the 
array will be multiplied by 255 and rounded.
\item array of numbers in any range - will be scaled or clipped depending 
on \code{scale} option. 
}

\item[\code{frame}] Request specific frame from multiframe (i.e., animated) GIF file. 
By default all frames are read from the file (\code{frame=0}). Setting  
\code{frame=1} will ensure that output is always a 2D matrix containing the 
first frame.  Some files have to be read frame by frame, for example: files 
with subimages of different sizes and files with both global and local 
color-maps (palettes).
\item[\code{col}] Color palette definition. Several formats are allowed: 
\Itemize{
\item array (list) of colors in the same format as output of palette 
functions.  Preferred format for precise color control.
\item palette function itself (ex. '\code{col=rainbow}'). Preferred 
format if not sure how many colors are needed.
\item character string with name of internally defined palette. At the 
moment only "gray" and "jet" (Matlab's jet palette) are defined.
\item character string with name of palette function  (ex. 
'\code{col="rainbow"}')
}
Usually palette will consist of 256 colors, which is the maximum allowed by 
GIF format. By default, grayscale will be used.
\item[\code{scale}] There are three approaches to rescaling the data to required 
[0, 255] integer range:
\Itemize{       
\item "smart" - Data is fitted to [0:255] range, only if it falls outside 
of it. Also, if \code{image} is an array of doubles in range [0, 1] than 
data is multiplied by 255.
\item "never" - Pixels with intensities outside of the allowed range are 
clipped to either 0 or 255. Warning is given. 
\item "always" - Data is always rescaled. If \code{image} is a array of 
doubles in range [0, 1] than data is multiplied by 255; otherwise it is 
scaled to fit to [0:255] range. 
}

\item[\code{delay}] In case of 3D arrays the data will be stored as animated GIF, and
\code{delay} controls speed of the animation. It is number of hundredths 
(1/100) of a second of delay between frames. 
\item[\code{comment}] Comments in text format are allowed in GIF files. Few file 
viewers can access them.
\item[\code{flip}] By default data is stored in the same orientation as data 
displayed by \code{\LinkA{print}{print}} function: row 1 is on top, image x-axis 
corresponds to columns and y-axis corresponds to rows. However function 
\code{\LinkA{image}{image}} adopted different standard: column 1 is on the bottom, 
image x-axis corresponds to rows and y-axis corresponds to columns. Set 
\code{flip} to \code{TRUE} to get the orientation used by \code{\LinkA{image}{image}}. 
\item[\code{transparent}] Optional color number to be shown as transparent. Has to be an
integer in [0:255] range. NA's in the \code{image} will be set to transparent.
\item[\code{interlace}] GIF files allow image rows to be \code{interlace}d, or 
reordered in such a way as to allow viewer to display image using 4 passes, 
making image sharper with each pass. Irrelevant feature on fast computers.
\item[\code{verbose}] Display details sections encountered while reading GIF file.
\end{ldescription}
\end{Arguments}
\begin{Details}\relax
Palettes often contain continuous colors, such that swapping palettes or 
rescaling of the image date does not affect image apperance in a drastic way. 
However, when working with non-continuous color-maps one should always provide 
image in [0:255] integer range (and set \code{scale="never"}), in order to 
prevent scaling.

If \code{NA} or other infinite numbers are found in the \code{image} by 
\code{write.gif}, they will be converted to numbers given by \code{transparent}.
If \code{transparent} color is not provided than it will be created, possibly 
after reshretching.

There are some GIF files not fully supported by \code{read.gif} function:
\Itemize{
\item "Plain Text Extension" is not supported, and will be ignored.
\item Multi-frame files with unique settings for each frame have to be read 
frame by frame. Possible settings include: frames with different sizes, 
frames using local color maps and frames using individual transparency colors.
}
\end{Details}
\begin{Value}
Function \code{write.gif} does not return anything.
Function \code{read.gif} returns a list with following fields:
\begin{ldescription}
\item[\code{image}] matrix or 3D array of integers in [0:255] range.
\item[\code{col}] color palette definitions with number of colors ranging from 1 
to 256. In case when \code{frame=0} only the first (usually global) 
color-map (palette) is returned.
\item[\code{comment}] Comments imbedded in GIF File
\item[\code{transparent}] color number corresponding to transparent color. If none 
was stated than NULL, otherwise an integer in [0:255] range. In order for 
\code{\LinkA{image}{image}} to display transparent colors correctly one
should use \code{y\$col[y\$transparent+1] = NA}. 
\end{ldescription}
\end{Value}
\begin{Author}\relax
Jarek Tuszynski (SAIC) \email{jaroslaw.w.tuszynski@saic.com}. 
Encoding Algorithm adapted from code by Christoph Hohmann, which was adapted 
from code by Michael Mayer. Parts of decoding algorithm adapted from code by 
David Koblas.
\end{Author}
\begin{References}\relax
Ziv, J., Lempel, A. (1977) \emph{An Universal Algorithm for Sequential Data 
Compression}, IEEE Transactions on Information Theory, May 1977. 

Copy of official file format description  
\url{http://www.danbbs.dk/\%7Edino/whirlgif/gif89.html}

Nicely explained file format description  
\url{http://semmix.pl/color/exgraf/eeg11.htm}

Christoph Hohmann code and documentation of encoding algorithm 
\url{http://members.aol.com/rf21exe/gif.htm}

Michael A, Mayer code \url{http://www.danbbs.dk/\%7Edino/whirlgif/gifcode.html}

Discussion of GIF file legal status can be found in 
\url{http://www.cloanto.com/users/mcb/19950127giflzw.html}.

Interesting page on one way of doing animations in R (with help of outside 
calls) can be found at
\url{http://pinard.progiciels-bpi.ca/plaisirs/animations/index.html}.
\end{References}
\begin{SeeAlso}\relax
Displaying of images can be done through functions: 
\code{\LinkA{image}{image}} (part of R),
\code{\LinkA{image.plot}{image.plot}} and \code{\LinkA{add.image}{add.image}} from 
\pkg{fields} or \code{\LinkA{plot.im}{plot.im}} from \pkg{spatstat} package, 
and possibly many other functions.

Displayed image can be saved in GIF, JPEG or PNG format using several 
different functions, like \code{\LinkA{HTMLplot}{HTMLplot}} from package \pkg{R2HTML}.

Functions for directly reading and writing image files: 
\Itemize{
\item \code{\LinkA{read.pnm}{read.pnm}} and \code{\LinkA{write.pnm}{write.pnm}} from 
\pkg{pixmap} package can process PBM, PGM and PPM images (file types 
supported by ImageMagic software)
\item \code{\LinkA{read.ENVI}{read.ENVI}} and \code{\LinkA{write.ENVI}{write.ENVI}} from this package
can process files in ENVI format. ENVI files can store 2D images and 3D data 
(multi-frame images), and are supported by most GIS (Geographic Information 
System) software including free "freelook".
}

There are many functions for creating and managing color palettes:
\Itemize{
\item \code{\LinkA{tim.colors}{tim.colors}} in package \pkg{fields} contains 
palette similar to Matlab's jet palette (see examples for simpler implementation) 
\item \code{\LinkA{rich.colors}{rich.colors}} in package \pkg{gplots} contains 
two palettes of continuous colors. 
\item Functions \code{\LinkA{brewer.pal}{brewer.pal}} from \pkg{RColorBrewer} 
package and \code{\LinkA{colorbrewer.palette}{colorbrewer.palette}} from \pkg{epitools} 
package contain tools for generating palettes
\item \code{\LinkA{rgb}{rgb}} and \code{\LinkA{hsv}{hsv}} 
creates palette from RGB or HSV 3-vectors. 
\item \code{\LinkA{col2rgb}{col2rgb}} translates 
palette colors to RGB 3-vectors. 
}
\end{SeeAlso}
\begin{Examples}
\begin{ExampleCode}
# visual comparison between image and plot
write.gif( volcano, "volcano.gif", col=terrain.colors, flip=TRUE, 
           scale="always", comment="Maunga Whau Volcano")
y = read.gif("volcano.gif", verbose=TRUE, flip=TRUE)
image(y$image, col=y$col, main=y$comment, asp=1)
# browseURL("file://volcano.gif")  # inspect GIF file on your hard disk

# test reading & writing
col = heat.colors(256) # choose colormap
trn = 222              # set transparent color
com = "Hello World"    # imbed comment in the file
write.gif( volcano, "volcano.gif", col=col, transparent=trn, comment=com)
y = read.gif("volcano.gif")
stopifnot(volcano==y$image, col==y$col, trn==y$transparent, com==y$comment)
# browseURL("file://volcano.gif") # inspect GIF file on your hard disk

# create simple animated GIF (using image function in a loop is very rough,
# but only way I know of displaying 'animation" in R)
x <- y <- seq(-4*pi, 4*pi, len=200)
r <- sqrt(outer(x^2, y^2, "+"))
image = array(0, c(200, 200, 10))
for(i in 1:10) image[,,i] = cos(r-(2*pi*i/10))/(r^.25)
write.gif(image, "wave.gif", col="rainbow")
y = read.gif("wave.gif")
for(i in 1:10) image(y$image[,,i], col=y$col, breaks=(0:256)-0.5, asp=1)
# browseURL("file://wave.gif") # inspect GIF file on your hard disk

# Another neat animation of Mandelbrot Set
jet.colors = colorRampPalette(c("#00007F", "blue", "#007FFF", "cyan", "#7FFF7F",
             "yellow", "#FF7F00", "red", "#7F0000")) # define "jet" palette
m = 400
C = complex( real=rep(seq(-1.8,0.6, length.out=m), each=m ), 
             imag=rep(seq(-1.2,1.2, length.out=m),      m ) )
C = matrix(C,m,m)
Z = 0
X = array(0, c(m,m,20))
for (k in 1:20) {
  Z = Z^2+C
  X[,,k] = exp(-abs(Z))
}
image(X[,,k], col=jet.colors(256))
write.gif(X, "Mandelbrot.gif", col=jet.colors, delay=100)
# browseURL("file://Mandelbrot.gif") # inspect GIF file on your hard disk
file.remove("wave.gif", "volcano.gif", "Mandelbrot.gif")

# Display interesting images from the web
## Not run: 
url = "http://www.ngdc.noaa.gov/seg/cdroms/ged_iib/datasets/b12/gifs/eccnv.gif"
y = read.gif(url, verbose=TRUE, flip=TRUE)
image(y$image, col=y$col, breaks=(0:length(y$col))-0.5, asp=1,
           main="January Potential Evapotranspiration mm/mo")
url = "http://www.ngdc.noaa.gov/seg/cdroms/ged_iib/datasets/b01/gifs/fvvcode.gif"
y = read.gif(url, flip=TRUE)
y$col[y$transparent+1] = NA # mark transparent color in R way
image(y$image, col=y$col[1:87], breaks=(0:87)-0.5, asp=1,
           main="Vegetation Types")
url = "http://talc.geo.umn.edu/people/grads/hasba002/erosion_vids/run2/r2_dems_5fps(8color).gif"
y = read.gif(url, verbose=TRUE, flip=TRUE)
for(i in 2:dim(y$image)[3]) 
  image(y$image[,,i], col=y$col, breaks=(0:length(y$col))-0.5,
            asp=1, main="Erosion in Drainage Basins")
## End(Not run)
\end{ExampleCode}
\end{Examples}

