\HeaderA{bitFlip}{Binary Flip (Not) Operator}{bitFlip}
\keyword{arith}{bitFlip}
\keyword{utilities}{bitFlip}
\begin{Description}\relax
The binary flip (not) operator, \code{bitFlip(a, w)}, \dQuote{flips every
bit} of \code{a} up to the \code{w}-th bit.
\end{Description}
\begin{Usage}
\begin{verbatim}
bitFlip(a, bitWidth=32)
\end{verbatim}
\end{Usage}
\begin{Arguments}
\begin{ldescription}
\item[\code{a}] numeric vector.
\item[\code{bitWidth}] scalar integer between 0 and 32.
\end{ldescription}
\end{Arguments}
\begin{Value}
binary numeric vector of the same length as \code{a} masked with
(2\textasciicircum{}\code{bitWidth})-1.  \code{\LinkA{NA}{NA}} is returned for any value of
\code{a} that is not finite or whose magnitude is greater or equal to
\eqn{2^{32}}{}.
\end{Value}
\begin{Author}\relax
Steve Dutky
\end{Author}
\begin{SeeAlso}\relax
\code{\LinkA{bitShiftL}{bitShiftL}}, \code{\LinkA{bitXor}{bitXor}}, etc.
\end{SeeAlso}
\begin{Examples}
\begin{ExampleCode}
 stopifnot(
  bitFlip(-1) == 0,
  bitFlip(0 ) == 2^32 - 1,
  bitFlip(0, bitWidth=8) == 255
 )
\end{ExampleCode}
\end{Examples}

